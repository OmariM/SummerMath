\documentclass[11pt,letterpaper,boxed]{hmcpset}
\usepackage{fullpage}
\setlength{\parskip}{6pt}
\setlength{\parindent}{0pt}
\usepackage[margin=1in]{geometry}
\usepackage{graphicx}
\usepackage{enumerate}
\usepackage{marvosym}
\usepackage{amssymb}
\usepackage{wasysym}
\usepackage{gensymb}
\usepackage{mathrsfs}
\usepackage{scrextend}
\usepackage{mathtools}
\usepackage{pgfplots}
\usepackage{xspace}
\usepackage[colorlinks]{hyperref}

\makeatletter
\renewcommand*\env@matrix[1][*\c@MaxMatrixCols c]{%
   \hskip -\arraycolsep
   \let\@ifnextchar\new@ifnextchar
   \array{#1}}
\makeatother

% --- style --- %
\renewcommand{\labelenumi}{{ (\alph{enumi})}}
\newcommand{\sand}{\quad \mbox{ and } \quad}
%\newcommand{\ds}{\displaystyle}
\allowdisplaybreaks

% --- making \xi look less awful --- %
\DeclareSymbolFont{CMletters}{OML}{cmm}{m}{it}
\DeclareMathSymbol{\xi}{\mathord}{CMletters}{"18}

% --- math --- %
\newcommand{\Z}{\mathbb{Z}}
\newcommand{\R}{\mathbb{R}}
\newcommand{\C}{\mathbb{C}}
\newcommand{\Q}{\mathbb{Q}}


\newcommand{\Lt}[1]{\mathcal{L}\crb{#1}}
\newcommand{\ilt}[1]{\mathcal{L}^{-1}\crb{#1}}

\newcommand{\pn}[1]{\left( #1 \right)}
\newcommand{\sqb}[1]{\left[ #1 \right]}
\newcommand{\crb}[1]{\left\{ #1 \right\}}
\newcommand{\lra}[1]{\left\langle #1 \right\rangle}
\newcommand{\magn}[1]{\left\lVert #1 \right\rVert}

\newcommand{\pdr}[2]{\frac{\partial #1}{\partial #2}}
\newcommand{\im}[1]{\text{im}\pn{#1}}
\newcommand{\m}[1]{\Z/#1\Z}

\newcommand{\VEC}[1]{\ensuremath{\mathbf{#1}}\xspace}
\DeclareMathOperator{\proj}{proj}
\newcommand{\vectorproj}[2][]{\proj_{\VEC{#1}}\VEC{#2}}

\newenvironment{amatrix}[1]{%
  \left(\begin{array}{@{}*{#1}{c}|c@{}}
}{%
  \end{array}\right)
}

\makeatletter
\renewcommand*\env@matrix[1][*\c@MaxMatrixCols c]{%
  \hskip -\arraycolsep
  \let\@ifnextchar\new@ifnextchar
  \array{#1}}
\makeatother

\newcommand{\spn}[1]{\text{span}\pn{#1}}

\newcommand*\Heq{\ensuremath{\overset{\kern2pt H}{=}}}

\name{Box \#$\rule{1cm}{0.15mm}$}
\class{Math 65 Section 1}
\assignment{Homework 2}
\duedate{16 May 2018}

\begin{document}

%\begin{center}
\noindent\textbf{Collaborators:} 
%\end{center} 

%\problemlist{}

\begin{problem}[Poole 6.2 \#30]
Let $\mathcal{B}$ be a set of vectors in a vector space $V$ with the property that
every vector in $V$ can be written uniquely as a linear combination of the vectors in $\mathcal{B}$. Prove that $\mathcal{B}$ is a basis for $V$.
\end{problem}

\begin{solution}
\vfill
\end{solution}
\newpage

\begin{problem}[Poole 6.2 \#36]
Find the dimension of the vector space $V$ and give a basis for $V$.
\[
	V = \crb{p(x) \text{ in } \mathcal{P}_2: xp'(x) = p(x)}
\]
\end{problem}

\begin{solution}
\vfill
\end{solution}
\newpage

\begin{problem}[Poole 6.2 \#40]
Find a formula for the dimension of the vector space
of symmetric $n \times n$ matrices.
\end{problem}

\begin{solution}
\vfill
\end{solution}
\newpage

\begin{problem}[Poole 6.2 \#42]
Let $U$ and $W$ be subspaces of a finite-dimensional
vector space $V$. Prove \textbf{Grassmann's Identity}:
\[
	\dim{\pn{U+W}}=\dim{U}+\dim{W}-\dim{\pn{U\cap W}}.
\]
[\textit{Hint:} The subspace of $U+W$ is defined in Exercise 48 of Section 6.1. Let
$\mathcal{B} = \crb{\VEC{v}_1,...,\VEC{v}_k}$ be a basis for $U \cap W$. Extend
$\mathcal{B}$ to a basis $\mathcal{C}$ of $U$ and a basis $\mathcal{D}$ of $W$. Prove
that $\mathcal{C}\cup\mathcal{D}$ is a basis for $U+W$.]
\end{problem}

\begin{solution}
\vfill
\end{solution}
\newpage

\begin{problem}[Poole 6.2 \#44]
Prove that the vector space $\mathcal{P}$ is infinite-dimensional. 
[\textit{Hint}: Suppose it has a finite basis. Show that there is some polynomial that is not a linear combination of this basis.]
\end{problem}

\begin{solution}
\vfill
\end{solution}
\newpage

\begin{problem}[Poole 6.2 \#58]
Let $\crb{\VEC{v}_1,..., \VEC{v}_n}$ be a basis for a vector space $V$. Prove that
\[
	\crb{\VEC{v}_1,\VEC{v}_1+\VEC{v}_2,\VEC{v}_1+\VEC{v}_2+\VEC{v}_3,...,\VEC{v}_1+...+\VEC{v}_n}
\]
is also a basis for $V$.
\end{problem}

\begin{solution}
\vfill
\end{solution}
\newpage

\begin{problem}[Poole 6.3 \#6]
\begin{enumerate}
\item Find the coordinate polynomials $p(x)_{\mathcal{B}}$ and $p(x)_{\mathcal{C}}$ of $p(x)$ with respect
to the bases $\mathcal{B}$ and $\mathcal{C}$, respectively.
\item Find the change-of-basis matrix $P_{\mathcal{C}\leftarrow\mathcal{B}}$ from $\mathcal{B}$ to $\mathcal{C}$.
\item Use your answer in part (b) to compute $p(x)_{\mathcal{C}}$ and compare your answer with the one found in part (a).
\item Find the change-of-basis matrix $P_{\mathcal{C}\leftarrow\mathcal{B}}$ from $\mathcal{C}$ to $\mathcal{B}$.
\item Use your answers to part (c) and (d) to compute $p(x)_{\mathcal{B}}$, and compare your answer with the one found in part (a).
\end{enumerate}
\[
	p(x) = 1 + 3x, \quad \mathcal{B} = \crb{1+x,1-x}, \quad \mathcal{C} = \crb{2x,4} \text{ in } \mathcal{P}_1.
\]
\end{problem}

\begin{solution}
\vfill
\end{solution}
\newpage

\begin{problem}[Poole 6.3 \#8]
\begin{enumerate}
\item Find the coordinate polynomials $p(x)_{\mathcal{B}}$ and $p(x)_{\mathcal{C}}$ of $p(x)$ with respect
to the bases $\mathcal{B}$ and $\mathcal{C}$, respectively.
\item Find the change-of-basis matrix $P_{\mathcal{C}\leftarrow\mathcal{B}}$ from $\mathcal{B}$ to $\mathcal{C}$.
\item Use your answer in part (b) to compute $p(x)_{\mathcal{C}}$ and compare your answer with the one found in part (a).
\item Find the change-of-basis matrix $P_{\mathcal{C}\leftarrow\mathcal{B}}$ from $\mathcal{C}$ to $\mathcal{B}$.
\item Use your answers to part (c) and (d) to compute $p(x)_{\mathcal{B}}$, and compare your answer with the one found in part (a).
\end{enumerate}
\[
	p(x) = 4-2x-x^2, \quad \mathcal{B} = \crb{x,1+x^2,x+x^2}, \quad \mathcal{C} = \crb{1, 1+x,x^2} \text{ in } \mathcal{P}_2.
\]
\end{problem}

\begin{solution}
\vfill
\end{solution}


\end{document}