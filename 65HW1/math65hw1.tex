\documentclass[11pt,letterpaper,boxed]{hmcpset}
\usepackage{fullpage}
\setlength{\parskip}{6pt}
\setlength{\parindent}{0pt}
\usepackage[margin=1in]{geometry}
\usepackage{graphicx}
\usepackage{enumerate}
\usepackage{marvosym}
\usepackage{amssymb}
\usepackage{wasysym}
\usepackage{gensymb}
\usepackage{mathrsfs}
\usepackage{scrextend}
\usepackage{mathtools}
\usepackage{pgfplots}
\usepackage{xspace}
\usepackage[colorlinks]{hyperref}

\makeatletter
\renewcommand*\env@matrix[1][*\c@MaxMatrixCols c]{%
   \hskip -\arraycolsep
   \let\@ifnextchar\new@ifnextchar
   \array{#1}}
\makeatother

% --- style --- %
\renewcommand{\labelenumi}{{ (\alph{enumi})}}
\newcommand{\sand}{\quad \mbox{ and } \quad}
%\newcommand{\ds}{\displaystyle}
\allowdisplaybreaks

% --- making \xi look less awful --- %
\DeclareSymbolFont{CMletters}{OML}{cmm}{m}{it}
\DeclareMathSymbol{\xi}{\mathord}{CMletters}{"18}

% --- math --- %
\newcommand{\Z}{\mathbb{Z}}
\newcommand{\R}{\mathbb{R}}
\newcommand{\C}{\mathbb{C}}
\newcommand{\Q}{\mathbb{Q}}


\newcommand{\Lt}[1]{\mathcal{L}\crb{#1}}
\newcommand{\ilt}[1]{\mathcal{L}^{-1}\crb{#1}}

\newcommand{\pn}[1]{\left( #1 \right)}
\newcommand{\sqb}[1]{\left[ #1 \right]}
\newcommand{\crb}[1]{\left\{ #1 \right\}}
\newcommand{\lra}[1]{\left\langle #1 \right\rangle}
\newcommand{\magn}[1]{\left\lVert #1 \right\rVert}

\newcommand{\pdr}[2]{\frac{\partial #1}{\partial #2}}
\newcommand{\im}[1]{\text{im}\pn{#1}}
\newcommand{\m}[1]{\Z/#1\Z}

\newcommand{\VEC}[1]{\ensuremath{\mathbf{#1}}\xspace}
\DeclareMathOperator{\proj}{proj}
\newcommand{\vectorproj}[2][]{\proj_{\VEC{#1}}\VEC{#2}}

\newenvironment{amatrix}[1]{%
  \left(\begin{array}{@{}*{#1}{c}|c@{}}
}{%
  \end{array}\right)
}

\makeatletter
\renewcommand*\env@matrix[1][*\c@MaxMatrixCols c]{%
  \hskip -\arraycolsep
  \let\@ifnextchar\new@ifnextchar
  \array{#1}}
\makeatother

\newcommand{\spn}[1]{\text{span}\pn{#1}}

\newcommand*\Heq{\ensuremath{\overset{\kern2pt H}{=}}}

\name{Box \#$\rule{1cm}{0.15mm}$}
\class{Math 65 Section 1}
\assignment{Homework 1}
\duedate{15 May 2018}

\begin{document}

%\begin{center}
\noindent\textbf{Collaborators:} 
%\end{center} 

%\problemlist{}

\begin{problem}[Poole 6.1 \#28]
Determine whether $W$ is a subspace of $V$.
\[
	V = M_{22}, \quad W = \crb{\begin{bmatrix}a & b\\b & 2a\end{bmatrix}}.
\]
\end{problem}

\begin{solution}
\vfill
\end{solution}
\newpage

\begin{problem}[Poole 6.1 \#34]
Determine whether $W$ is a subspace of $V$.
\[
	V = \mathscr{P}_2, \quad W = \crb{bx+cx^2}.
\]
\end{problem}

\begin{solution}
\vfill
\end{solution}
\newpage

\begin{problem}[Poole 6.1 \#36]
Determine whether $W$ is a subspace of $V$.
\[
	V = \mathscr{P}_2, \quad W = \crb{a+bx+cx^2: abc=0}.
\]
\end{problem}

\begin{solution}
\vfill
\end{solution}
\newpage

\begin{problem}[Poole 6.1 \#38]
Determine whether $W$ is a subspace of $V$.
\[
	V = \mathscr{F}, \quad W = \crb{f \text{ in } \mathscr{F}: f(-x)=f(x)}.
\]
\end{problem}

\begin{solution}
\vfill
\end{solution}
\newpage

\begin{problem}[Poole 6.1 \#46]
Let $V$ be a vector space with subspaces $U$ and $W$. Prove that $U \cap W$ is a subspace of $V$.
\end{problem}

\begin{solution}
\vfill
\end{solution}
\newpage

\begin{problem}[Poole 6.1 \#48]
Let $V$ be a vector space with subspaces $U$ and $W$. Define the \textbf{sum of} $\mathbf{U}$ \textbf{and} $\mathbf{W}$ to be
\[
	U+W = \crb{\VEC{u} + \VEC{w}: \VEC{u} \text{ is in } U, \VEC{w} \text{ is in } W}
\]
\begin{enumerate}
\item[(a)] If $V = \R^3$, $U$ is the $x$-axis, and $W$ is the $y$-axis, what is $U+W$?
\item[(b)] If $U$ and $W$ are subspaces of a vector space $V$, prove that $U+W$ is a subspace of $V$.
\end{enumerate}
\end{problem}

\begin{solution}
\vfill
\end{solution}
\newpage

\begin{problem}[Poole 6.1 \#60]
Is $M_{22}$ spanned by $\begin{bmatrix}1 & 0\\1&0\end{bmatrix}, \begin{bmatrix}1 & 1\\1&0\end{bmatrix}, \begin{bmatrix}1 & 1\\1&1\end{bmatrix},
 \begin{bmatrix}0 & -1\\1&0\end{bmatrix}?$
\end{problem}

\begin{solution}
\vfill
\end{solution}
\newpage

\begin{problem}[Poole 6.1 \#62]
Is $\mathscr{P}_2$ spanned by $1+x+2x^2, 2+x+2x^2, -1+x+2x^2$?
\end{problem}

\begin{solution}
\vfill
\end{solution}
\newpage


\end{document}