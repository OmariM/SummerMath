\documentclass[11pt,letterpaper,boxed]{hmcpset}
\usepackage{fullpage}
\setlength{\parskip}{6pt}
\setlength{\parindent}{0pt}
\usepackage[margin=1in]{geometry}
\usepackage{graphicx}
\usepackage{enumerate}
\usepackage{marvosym}
\usepackage{amssymb}
\usepackage{wasysym}
\usepackage{gensymb}
\usepackage[inline]{enumitem}
\usepackage{mathrsfs}
\usepackage{scrextend}
\usepackage{mathtools}
\usepackage{pgfplots}
\usepackage{xspace}
\usepackage[colorlinks]{hyperref}

\makeatletter
\renewcommand*\env@matrix[1][*\c@MaxMatrixCols c]{%
   \hskip -\arraycolsep
   \let\@ifnextchar\new@ifnextchar
   \array{#1}}
\makeatother

% --- style --- %
\renewcommand{\labelenumi}{{ (\alph{enumi})}}
\newcommand{\sand}{\quad \mbox{ and } \quad}
%\newcommand{\ds}{\displaystyle}
\allowdisplaybreaks

% --- making \xi look less awful --- %
\DeclareSymbolFont{CMletters}{OML}{cmm}{m}{it}
\DeclareMathSymbol{\xi}{\mathord}{CMletters}{"18}

% --- math --- %
\newcommand{\Z}{\mathbb{Z}}
\newcommand{\R}{\mathbb{R}}
\newcommand{\C}{\mathbb{C}}
\newcommand{\Q}{\mathbb{Q}}


\newcommand{\Lt}[1]{\mathcal{L}\crb{#1}}
\newcommand{\ilt}[1]{\mathcal{L}^{-1}\crb{#1}}

\newcommand{\pn}[1]{\left( #1 \right)}
\newcommand{\sqb}[1]{\left[ #1 \right]}
\newcommand{\crb}[1]{\left\{ #1 \right\}}
\newcommand{\lra}[1]{\left\langle #1 \right\rangle}
\newcommand{\magn}[1]{\left\lVert #1 \right\rVert}

\newcommand{\pdr}[2]{\frac{\partial #1}{\partial #2}}
\newcommand{\im}[1]{\text{im}\pn{#1}}
\newcommand{\m}[1]{\Z/#1\Z}

\newcommand{\VEC}[1]{\ensuremath{\mathbf{#1}}\xspace}
\DeclareMathOperator{\proj}{proj}
\newcommand{\vectorproj}[2][]{\proj_{\VEC{#1}}\VEC{#2}}

\newenvironment{amatrix}[1]{%
  \left(\begin{array}{@{}*{#1}{c}|c@{}}
}{%
  \end{array}\right)
}

\makeatletter
\renewcommand*\env@matrix[1][*\c@MaxMatrixCols c]{%
  \hskip -\arraycolsep
  \let\@ifnextchar\new@ifnextchar
  \array{#1}}
\makeatother

\newcommand{\spn}[1]{\text{span}\pn{#1}}

\newcommand*\Heq{\ensuremath{\overset{\kern2pt H}{=}}}

\name{Box \#$\rule{1cm}{0.15mm}$}
\class{Math 65 Section 1}
\assignment{Homework 3}
\duedate{17 May 2018}

\begin{document}

%\begin{center}
\noindent\textbf{Collaborators:} 
%\end{center} 

%\problemlist{}

\begin{problem}[Poole 6.3 \#16]
Let $\mathcal{B}$ and $\mathcal{C}$ be bases for $\mathcal{P}_2$. If $\mathcal{B}=\crb{x,1+x,1-x+x^2}$ and
the change-of-basis matrix from $\mathcal{B}$ to $\mathcal{C}$ is
\[
	\mathcal{P}_{\mathcal{C}\leftarrow\mathcal{B}} = \begin{bmatrix}1 & 0 & 0\\0 & 2&1\\-1&1&1\end{bmatrix}.
\]
Find $\mathcal{C}$.
\end{problem}

\begin{solution}
\vfill
\end{solution}
\newpage

\begin{problem}[Poole 6.3 \#22]
Let $V$ be an $n$-dimensional vector space with basis $\mathcal{B}=\crb{\VEC{v}_1,...,\VEC{v}_n}$. Let
$P$ be an invertible $n \times n$ matrix and set
\[
	\VEC{u}_i = p_{1i}\VEC{v}_1+...+p_{ni}\VEC{v}_n
\]
for $i=1,...,n$. Prove that $\mathcal{C} = \crb{\VEC{u}_1,...,\VEC{u}_n}$ is a basis for $V$	 and show that
$P = P_{\mathcal{B}\leftarrow\mathcal{C}}$.
\end{problem}

\begin{solution}
\vfill
\end{solution}
\newpage

\begin{problem}[Poole 6.4 \#20]
Show that there is no linear transformation $T: \R^3 \rightarrow \mathcal{P}_2$ such that
\[
	T\begin{bmatrix}2\\1\\0\end{bmatrix}=1+x, \quad T\begin{bmatrix}3\\0\\2\end{bmatrix}=2-x+x^2, \quad T\begin{bmatrix}0\\6\\-8\end{bmatrix}=-2+2x^2.
\]
\end{problem}

\begin{solution}
\vfill
\end{solution}
\newpage

\begin{problem}[Poole 6.4 \#24]
Let $\VEC{v}_1,...,\VEC{v}_n$ be vectors in a vector space $V$ and let $T: V \rightarrow W$ be a linear transformation.
\begin{enumerate}
\item If $\crb{T\pn{\VEC{v}_1},...,T\pn{\VEC{v}_n}}$ is linearly independent in $W$, show that $\crb{\VEC{v}_1,...,\VEC{v}_n}$ is
linearly independent in $V$.
\item Show that the converse of part (a) is false. That is, it is not necessarily true that 
if $\crb{\VEC{v}_1,...,\VEC{v}_n}$ is linearly independent in $V$, then $\crb{T\pn{\VEC{v}_1},..., T\pn{\VEC{v}_n}}$ is linearly independent in $W$. 
Illustrate this with an example $T: \R^2 \rightarrow \R^2.$
\end{enumerate}
\end{problem}

\begin{solution}
\vfill
\end{solution}
\newpage

\begin{problem}[Poole 6.4 \#32]
Let $T: V \rightarrow V$ be a linear transformation such that $T \circ T = I$.
\begin{enumerate}
\item Show that $\crb{\VEC{v},T(\VEC{v})}$ is linearly dependent if and only if $T(\VEC{v})=\pm\VEC{v}$.
\item Give an example of such a linear transformation with $V = \R^2$.
\end{enumerate}
\end{problem}

\begin{solution}
\vfill
\end{solution}
\newpage

\begin{problem}[Poole 6.5 \#4]
Let $T: \mathcal{P}_2 \rightarrow \mathcal{P}_2$ be the linear transformation defined by
$T(p(x)) = xp'(x)$.
\begin{enumerate}
\item Which, if any, of the following are in ker$(T)$?\\
	\begin{enumerate*}
	\item[(i)] $1$ \item[(ii)] $x$ \item[(iii)] $x^2$
	\end{enumerate*}
\item Which, if any, of the polynomials in part (a) are in range($T$)?
\item Describe ker($T$) and range($T$).
\end{enumerate}
\end{problem}

\begin{solution}
\vfill
\end{solution}
\newpage

\begin{problem}[Poole 6.5 \#28]
Show that $T: \mathcal{P}_n \rightarrow \mathcal{P}_n$ defined by $T(p(x)) = p(x-2)$ is an isomorphism.
\end{problem}

\begin{solution}
\vfill
\end{solution}
\newpage

\begin{problem}[Poole 6.5 \#34]
Let $S: V\rightarrow W$ and $T: U \rightarrow V$ be linear transformations.
\begin{enumerate}
\item Prove that if $S \circ T$ is one-to-one, so is $T$.
\item Prove that if $S \circ T$ is onto, so is $S$.
\end{enumerate}
\end{problem}

\begin{solution}
\vfill
\end{solution}


\end{document}