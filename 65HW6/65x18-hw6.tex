\documentclass[11pt,letterpaper,boxed]{hmcpset}
\usepackage{fullpage}
\setlength{\parskip}{6pt}
\setlength{\parindent}{0pt}
\usepackage[margin=1in]{geometry}
\usepackage{graphicx}
\usepackage{enumerate}
\usepackage{marvosym}
\usepackage{amssymb}
\usepackage{wasysym}
\usepackage{gensymb}
\usepackage{mathrsfs}
\usepackage{scrextend}
\usepackage{mathtools}
\usepackage{pgfplots}
\usepackage{xspace}
\usepackage[colorlinks]{hyperref}

\makeatletter
\renewcommand*\env@matrix[1][*\c@MaxMatrixCols c]{%
   \hskip -\arraycolsep
   \let\@ifnextchar\new@ifnextchar
   \array{#1}}
\makeatother

% --- style --- %
\renewcommand{\labelenumi}{{ (\alph{enumi})}}
\newcommand{\sand}{\quad \mbox{ and } \quad}
%\newcommand{\ds}{\displaystyle}
\allowdisplaybreaks

% --- making \xi look less awful --- %
\DeclareSymbolFont{CMletters}{OML}{cmm}{m}{it}
\DeclareMathSymbol{\xi}{\mathord}{CMletters}{"18}

% --- math --- %
\newcommand{\Z}{\mathbb{Z}}
\newcommand{\R}{\mathbb{R}}
\newcommand{\C}{\mathbb{C}}
\newcommand{\Q}{\mathbb{Q}}


\newcommand{\Lt}[1]{\mathcal{L}\crb{#1}}
\newcommand{\ilt}[1]{\mathcal{L}^{-1}\crb{#1}}

\newcommand{\pn}[1]{\left( #1 \right)}
\newcommand{\sqb}[1]{\left[ #1 \right]}
\newcommand{\crb}[1]{\left\{ #1 \right\}}
\newcommand{\lra}[1]{\left\langle #1 \right\rangle}
\newcommand{\magn}[1]{\left\lVert #1 \right\rVert}

\newcommand{\pdr}[2]{\frac{\partial #1}{\partial #2}}
\newcommand{\im}[1]{\text{im}\pn{#1}}
\newcommand{\m}[1]{\Z/#1\Z}

\newcommand{\VEC}[1]{\ensuremath{\mathbf{#1}}\xspace}
\DeclareMathOperator{\proj}{proj}
\newcommand{\vectorproj}[2][]{\proj_{\VEC{#1}}\VEC{#2}}

\newenvironment{amatrix}[1]{%
  \left(\begin{array}{@{}*{#1}{c}|c@{}}
}{%
  \end{array}\right)
}

\makeatletter
\renewcommand*\env@matrix[1][*\c@MaxMatrixCols c]{%
  \hskip -\arraycolsep
  \let\@ifnextchar\new@ifnextchar
  \array{#1}}
\makeatother

\newcommand{\spn}[1]{\text{span}\pn{#1}}

\newcommand*\Heq{\ensuremath{\overset{\kern2pt H}{=}}}

\name{Box \#$\rule{1cm}{0.15mm}$}
\class{Math 65 Section 1}
\assignment{Homework 6}
\duedate{24 May 2018}

\begin{document}

%\begin{center}
\noindent\textbf{Collaborators:} 
%\end{center} 

%\problemlist{}


\begin{problem}[1.] 
The general linear $n^{\text{th}}$-order constant-coefficient homogeneous DE for a function $u(t)$ has the form
\begin{equation}
a_n u^{(n)} + a_{n-1} u^{(n-1)} + a_2 u'' + a_1 u' + a_0 u = 0
\end{equation}
Write this as a system of first-order ODEs, in matrix form.
\end{problem}

\begin{solution}
\vfill
\end{solution}
\newpage

\begin{problem}[2.] 
Find the general solution of $\mathbf{x} \, ' =  \begin{bmatrix} 2 & 1 \\ -3 & 6 \end{bmatrix} \mathbf{x}.$ Express your answer in the form $\mathbf{x}(t) = \Psi(t) \mathbf{c}$ where $\Psi(t)$ is a fundamental matrix.
\end{problem}

\begin{solution}
\vfill
\end{solution}
\newpage

\begin{problem}[3.]
 Find the general solution for the linear system
\begin{equation}
 \mathbf{x} \, ' = \begin{bmatrix} 2 & 2 & 1 \\ 1 & 3 & 1 \\ 1 & 2 &2 \end{bmatrix} \mathbf{x}.
 \end{equation}
\end{problem}

\begin{solution}
\vfill
\end{solution}
\newpage

\begin{problem}[4.]
Suppose  $\lambda \in \C$  and  $\mathbf{v} \in \C^n$. Determine the real and imaginary parts of the  function
\[  \mathbf{x}(t) = e^{\lambda t} \mathbf{v}.  \]
For consistency in grading, assume $\lambda = \alpha + i \beta$   ($\alpha,\beta \in \R$) and $\mathbf{v} = \mathbf{p} + i \mathbf{q}$  ($\mathbf{p},\mathbf{q} \in \R^n$). 
\end{problem}

\begin{solution}
\vfill
\end{solution}
\newpage

\begin{problem}[5.] Find the eigenvalues and eigenvectors of the matrix $A = \begin{bmatrix} 2 & - 1 \\ 1 & 2 \end{bmatrix}$. If the eigenvalues are complex, express your eigenvectors in the form $\mathbf{p} + i \mathbf{q}$, where $\mathbf{p}, \mathbf{q} \in \R^2$.
\end{problem}

\begin{solution}
\vfill
\end{solution}
\newpage

\begin{problem}[6.]  
 Find the general real-valued solution for the system
\begin{equation}
\mathbf{x} \, ' = \begin{bmatrix} 2 & -1 \\ 1 & 2 \end{bmatrix} \mathbf{x}.
 \end{equation}
Express your answer in the form $\mathbf{x}(t) = \Psi(t) \mathbf{c}$ where $\Psi(t)$ is a fundamental matrix.
 \end{problem}
 
 \begin{solution}
\vfill
\end{solution}
\newpage  

\begin{problem}[7.] 
Solve the initial value problem 
\begin{equation}
\mathbf{x} \, ' = \begin{bmatrix} 0 & 1 \\ -2 & -2 \end{bmatrix} \mathbf{x}
\end{equation}
with initial condition $\mathbf{x}(0) =  \begin{bmatrix}  -1 \\ 1 \end{bmatrix}$ and determine $\displaystyle \lim_{t \to \infty}  \mathbf{x}(t)$ for your solution.
\end{problem}

\begin{solution}
\vfill
\end{solution}
\newpage

\begin{problem}[8.]
Consider the inhomogeneous linear system
\begin{equation}
\mathbf{x} \,' = A \, \mathbf{x} + \mathbf{g}(t)
\label{ls2}
\end{equation}
where $A \in M_{nn}(\R)$ and $\mathbf{g}:\R \to \R^n$.  Suppose $\mathbf{x}_p$ is a particular solution of the inhomogeneous linear system and  $\{  \mathbf{x}_1, \ldots, \mathbf{x}_n \}$ is a basis for the space of solutions of the associated homogeneous system $\mathbf{x} ' = A \, \mathbf{x}$. Prove that if  $\mathbf{y}$ is any solution of the inhomogeneous equation \eqref{ls2} then there must exist constants $c_1, \ldots, c_n$ such that 
\[ \mathbf{y}(t) = c_1  \mathbf{x}_1(t) + \cdots +  c_n  \mathbf{x}_n(t)  +   \mathbf{x}_p(t).   \]
Therefore, if you know \textit{one} solution to the inhomogeneous problem you know them \textit{all}, up to something in the homogeneous solution space. 
  \end{problem}
  
  \begin{solution}
\vfill
\end{solution}
\newpage

\end{document}







