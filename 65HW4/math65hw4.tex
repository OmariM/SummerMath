\documentclass[11pt,letterpaper,boxed]{hmcpset}
\usepackage{fullpage}
\setlength{\parskip}{6pt}
\setlength{\parindent}{0pt}
\usepackage[margin=1in]{geometry}
\usepackage{graphicx}
\usepackage{enumerate}
\usepackage{marvosym}
\usepackage{amssymb}
\usepackage{wasysym}
\usepackage{gensymb}
\usepackage{mathrsfs}
\usepackage{scrextend}
\usepackage{mathtools}
\usepackage{pgfplots}
\usepackage{xspace}
\usepackage[colorlinks]{hyperref}

\makeatletter
\renewcommand*\env@matrix[1][*\c@MaxMatrixCols c]{%
   \hskip -\arraycolsep
   \let\@ifnextchar\new@ifnextchar
   \array{#1}}
\makeatother

% --- style --- %
\renewcommand{\labelenumi}{{ (\alph{enumi})}}
\newcommand{\sand}{\quad \mbox{ and } \quad}
%\newcommand{\ds}{\displaystyle}
\allowdisplaybreaks

% --- making \xi look less awful --- %
\DeclareSymbolFont{CMletters}{OML}{cmm}{m}{it}
\DeclareMathSymbol{\xi}{\mathord}{CMletters}{"18}

% --- math --- %
\newcommand{\Z}{\mathbb{Z}}
\newcommand{\R}{\mathbb{R}}
\newcommand{\C}{\mathbb{C}}
\newcommand{\Q}{\mathbb{Q}}


\newcommand{\Lt}[1]{\mathcal{L}\crb{#1}}
\newcommand{\ilt}[1]{\mathcal{L}^{-1}\crb{#1}}

\newcommand{\pn}[1]{\left( #1 \right)}
\newcommand{\sqb}[1]{\left[ #1 \right]}
\newcommand{\crb}[1]{\left\{ #1 \right\}}
\newcommand{\lra}[1]{\left\langle #1 \right\rangle}
\newcommand{\magn}[1]{\left\lVert #1 \right\rVert}

\newcommand{\pdr}[2]{\frac{\partial #1}{\partial #2}}
\newcommand{\im}[1]{\text{im}\pn{#1}}
\newcommand{\m}[1]{\Z/#1\Z}

\newcommand{\VEC}[1]{\ensuremath{\mathbf{#1}}\xspace}
\DeclareMathOperator{\proj}{proj}
\newcommand{\vectorproj}[2][]{\proj_{\VEC{#1}}\VEC{#2}}

\newenvironment{amatrix}[1]{%
  \left(\begin{array}{@{}*{#1}{c}|c@{}}
}{%
  \end{array}\right)
}

\makeatletter
\renewcommand*\env@matrix[1][*\c@MaxMatrixCols c]{%
  \hskip -\arraycolsep
  \let\@ifnextchar\new@ifnextchar
  \array{#1}}
\makeatother

\newcommand{\spn}[1]{\text{span}\pn{#1}}

\newcommand*\Heq{\ensuremath{\overset{\kern2pt H}{=}}}

\name{Box \#$\rule{1cm}{0.15mm}$}
\class{Math 65 Section 1}
\assignment{Homework 4}
\duedate{18 May 2018}

\begin{document}

%\begin{center}
\noindent\textbf{Collaborators:} 
%\end{center} 

%\problemlist{}

\begin{problem}[Poole 6.6 \#2]
Find the matrix $\begin{bmatrix}T\end{bmatrix}_{\mathcal{C}\leftarrow\mathcal{B}}$ of the linear transformation
$T: V \rightarrow W$ with respect to the bases $\mathcal{B}$ and $\mathcal{C}$ of $V$ and $W$, respectively. Verify
Theorem 2.26 for the vector $\VEC{v}$ by computing $T(\vec{v})$ directly and using the theorem.\\

$T:\mathcal{P}_1\rightarrow \mathcal{P}_1$ defined by $T(a+bx)=b-ax$, $\mathcal{B} = \crb{1+x,1-x}$, $\mathcal{C} = \crb{1,x}$, $\VEC{v} = p(x) = 4+2x$.
\end{problem}

\begin{solution}
\vfill
\end{solution}
\newpage

\begin{problem}[Poole 6.6 \#4]
Find the matrix $\begin{bmatrix}T\end{bmatrix}_{\mathcal{C}\leftarrow\mathcal{B}}$ of the linear transformation
$T: V \rightarrow W$ with respect to the bases $\mathcal{B}$ and $\mathcal{C}$ of $V$ and $W$, respectively. Verify
Theorem 2.26 for the vector $\VEC{v}$ by computing $T(\vec{v})$ directly and using the theorem.\\

$T:\mathcal{P}_2\rightarrow \mathcal{P}_2$ defined by $T(p(x))=p(x+2)$, $\mathcal{B} = \crb{1,x+2,(x+2)^2}$, $\mathcal{C} = \crb{1,x,x^2}$, $\VEC{v} = p(x) = a+bx+cx^2$.
\end{problem}

\begin{solution}
\vfill
\end{solution}
\newpage

\begin{problem}[Poole 6.6 \#12]
Find the matrix $\begin{bmatrix}T\end{bmatrix}_{\mathcal{C}\leftarrow\mathcal{B}}$ of the linear transformation
$T: V \rightarrow W$ with respect to the bases $\mathcal{B}$ and $\mathcal{C}$ of $V$ and $W$, respectively. Verify
Theorem 2.26 for the vector $\VEC{v}$ by computing $T(\vec{v})$ directly and using the theorem.\\

$T: M_{22} \rightarrow M_{22}$ defined by $T(A) = A - A^T$, $\mathcal{B} = \mathcal{C} = \crb{E_{11},E_{12},E_{21},E_{22}}$, $\VEC{v} = A = \begin{bmatrix}a&b\\c&d\end{bmatrix}$.
\end{problem}

\begin{solution}
\vfill
\end{solution}
\newpage

\begin{problem}[Poole 6.6 \#14]
Consider the subspace $W$ of $\mathcal{D}$, given by $W = \spn{e^{2x},e^{-2x}}$.
\begin{enumerate}
\item Show that the differential operator $D$ maps $W$ into itself.
\item Find the matrix of $D$ with respect to $\mathcal{B} = \crb{e^{2x},e^{-2x}}$.
\item Compute the derivative of $f(x) = e^{2x}-3e^{-2x}$ indirectly, using Theorem 6.26,
and verify that it agrees with $f'(x)$ as computed directly.
\end{enumerate}
\end{problem}

\begin{solution}
\vfill
\end{solution}
\newpage

\begin{problem}[Poole 6.6 \#18]
$T:U\rightarrow V$ and $S: V\rightarrow W$ are linear transformation and $\mathcal{B}$, $\mathcal{C}$, and $\mathcal{D}$ are bases for $U, V,$ and $W$, respectively. Compute
$[S\circ T]_{\mathcal{D}\leftarrow\mathcal{B}}$ in two ways:
\begin{enumerate}
\item By finding $S\circ T$ directly and then computing its matrix 
\item by finding the matrices of $S$ and $T$ separately and using Theorem 6.27.
\end{enumerate}
$T:\mathcal{P}_1\rightarrow\mathcal{P}_2$ defined by $T(p(x)) = p(x+1)$, $S:\mathcal{P}_2\rightarrow\mathcal{P}_2$ defined by $S(p(x)) = p(x+1)$,
$\mathcal{B} = \crb{1,x}$, $\mathcal{C} = \mathcal{D} = \crb{1,x,x^2}$.
\end{problem}

\begin{solution}
\vfill
\end{solution}
\newpage

\begin{problem}[Poole 6.6 \#22]
Determine whether the linear transfor�mation $T$ is invertible by considering its matrix with respect 
to the standard bases. If $T$ is invertible, use Theorem 6.28 and the method of Example 6.82 to find $T^{-1}$.
\[
	T: \mathcal{P}_2 \rightarrow \mathcal{P}_2 \text{ defined by } T(p(x)) = p'(x).
\]
\end{problem}

\begin{solution}
\vfill
\end{solution}
\newpage

\begin{problem}[Poole 6.6 \#32]
a linear transformation $T : V \rightarrow V$ is given. If possible, find a basis $\mathcal{C}$ for $V$ such that the matrix
$[T]_{\mathcal{C}}$ of $T$ with respect to $C$ is diagonal.
\[
	T: \R^2 \rightarrow \R^2 \text{ defined by } T\begin{bmatrix}a\\b\end{bmatrix} = \begin{bmatrix}a-b\\a+b\end{bmatrix}.
\]
\end{problem}

\begin{solution}
\vfill
\end{solution}
\newpage

\begin{problem}[Poole 6.6 \#34]
a linear transformation $T : V \rightarrow V$ is given. If possible, find a basis $\mathcal{C}$ for $V$ such that the matrix
$[T]_{\mathcal{C}}$ of $T$ with respect to $C$ is diagonal.
\[
	T: \mathcal{P}_2 \rightarrow \mathcal{P}_2 \text{ defined by } T\begin{bmatrix}a\\b\end{bmatrix} = \begin{bmatrix}a-b\\a+b\end{bmatrix}.
\]
\end{problem}

\begin{solution}
\vfill
\end{solution}


\end{document}