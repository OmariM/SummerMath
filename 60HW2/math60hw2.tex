\documentclass[11pt,letterpaper,boxed]{hmcpset}
\usepackage{fullpage}
\setlength{\parskip}{6pt}
\setlength{\parindent}{0pt}
\usepackage[margin=1in]{geometry}
\usepackage{graphicx}
\usepackage{enumerate}
\usepackage{marvosym}
\usepackage{amssymb}
\usepackage{wasysym}
\usepackage{gensymb}
\usepackage{mathrsfs}
\usepackage{scrextend}
\usepackage{mathtools}
\usepackage{pgfplots}
\usepackage{xspace}
\usepackage[colorlinks]{hyperref}

\makeatletter
\renewcommand*\env@matrix[1][*\c@MaxMatrixCols c]{%
   \hskip -\arraycolsep
   \let\@ifnextchar\new@ifnextchar
   \array{#1}}
\makeatother

% --- style --- %
\renewcommand{\labelenumi}{{ (\alph{enumi})}}
\newcommand{\sand}{\quad \mbox{ and } \quad}
%\newcommand{\ds}{\displaystyle}
\allowdisplaybreaks

% --- making \xi look less awful --- %
\DeclareSymbolFont{CMletters}{OML}{cmm}{m}{it}
\DeclareMathSymbol{\xi}{\mathord}{CMletters}{"18}

% --- math --- %
\newcommand{\Z}{\mathbb{Z}}
\newcommand{\R}{\mathbb{R}}
\newcommand{\C}{\mathbb{C}}
\newcommand{\Q}{\mathbb{Q}}


\newcommand{\Lt}[1]{\mathcal{L}\crb{#1}}
\newcommand{\ilt}[1]{\mathcal{L}^{-1}\crb{#1}}

\newcommand{\pn}[1]{\left( #1 \right)}
\newcommand{\sqb}[1]{\left[ #1 \right]}
\newcommand{\crb}[1]{\left\{ #1 \right\}}
\newcommand{\lra}[1]{\left\langle #1 \right\rangle}
\newcommand{\magn}[1]{\left\lVert #1 \right\rVert}

\newcommand{\pdr}[2]{\frac{\partial #1}{\partial #2}}
\newcommand{\pdrr}[2]{\frac{\partial^2 #1}{\partial #2^2}}
\newcommand{\im}[1]{\text{im}\pn{#1}}
\newcommand{\m}[1]{\Z/#1\Z}

\newcommand{\VEC}[1]{\ensuremath{\mathbf{#1}}\xspace}
\DeclareMathOperator{\proj}{proj}
\newcommand{\vectorproj}[2][]{\proj_{\VEC{#1}}\VEC{#2}}

\newenvironment{amatrix}[1]{%
  \left(\begin{array}{@{}*{#1}{c}|c@{}}
}{%
  \end{array}\right)
}

\makeatletter
\renewcommand*\env@matrix[1][*\c@MaxMatrixCols c]{%
  \hskip -\arraycolsep
  \let\@ifnextchar\new@ifnextchar
  \array{#1}}
\makeatother

\newcommand{\spn}[1]{\text{span}\pn{#1}}

\newcommand*\Heq{\ensuremath{\overset{\kern2pt H}{=}}}

\name{Box \#$\rule{1cm}{0.15mm}$}
\class{Math 60 Section 1}
\assignment{Homework 2}
\duedate{16 May 2018}

\begin{document}

%\begin{center}
\noindent\textbf{Collaborators:} 
%\end{center} 

%\problemlist{}

\begin{problem}[Colley 2.3 \#22]
Find the gradient $\nabla f(\mathbf{a})$ where
\[
	f(x,y) = e^{xy}+\ln(x-y), \quad \mathbf{a}=(2,1).
\]
\end{problem}

\begin{solution}
\vfill
\end{solution}
\newpage

\begin{problem}[Colley 2.3 \#33]
Find the matrix $D\mathbf{f}(\mathbf{a})$ of partial derivatives, where
\[
	\mathbf{f}(s,t) = (s^2,st,t^2),\quad \mathbf{a} = (-1,1).
\]
\end{problem}

\begin{solution}
\vfill
\end{solution}
\newpage

\begin{problem}[Colley 2.3 \#38]
Find an equation for the plane tangent to the graph of $z=4\cos{(xy)}$ at
the point $(\pi/3,1,2)$.
\end{problem}

\begin{solution}
\vfill
\end{solution}
\newpage

\begin{problem}[Colley 2.3 \#42]
Suppose that you have the following information concerning a differentiable function $f$:
\[
	f(2,3) = 12, \quad f(1.98,3) = 12.1, \quad f(2,3.01) = 12.2.
\]
\begin{enumerate}
\item[(a)] Give an approximate equation for the plane tangent to the graph of $f$ at $(2,3,12)$.
\item[(b)] Use the result of part (a) to estimate $f(1.98, 2.98)$.
\end{enumerate}
\end{problem}

\begin{solution}
\vfill
\end{solution}
\newpage

\begin{problem}[Colley 2.4 \#2]
Verify the sum rule for derivative matrices for the following pair of functions.
\[
	\mathbf{f}(x,y) = (e^{x+y},xe^y), \quad \mathbf{g}(x,y)=(\ln{(xy)},ye^x).
\]
\end{problem}

\begin{solution}
\vfill
\end{solution}
\newpage

\begin{problem}[Colley 2.4 \#14]
For the following function, determine all second-order partial derivatives (including mixed partials).
\[
	f(x,y) = e^{{x^2}+{y^2}}.
\]
\end{problem}

\begin{solution}
\vfill
\end{solution}
\newpage

\begin{problem}[Colley 2.4 \#22]
Consider the function $F(x,y,z)=2x^3y+xz^2+y^3z^5-7xyz.$
\begin{enumerate}
\item[(a)] Find $F_{xx},F_{yy},F_{zz}$.
\item[(b)] Calculate the mixed second-order partials $F_{xy}, F_{yx}, F_{xz},
F_{zx}, F_{yz}, F_{zy},$ and verify Theorem 4.3.
\end{enumerate}
\end{problem}

\begin{solution}
\vfill
\end{solution}
\newpage

\begin{problem}[Colley 2.4 \#29a]
The three-dimensional \textbf{heat equation} is the partial differential equation
\[
	k\pn{\pdrr{T}{x}+\pdrr{T}{y}+\pdrr{T}{z}} = \pdr{T}{t},
\]
where $k$ is a positive constant. It models the temperature $T(x,y,z,t)$ at the point $(x,y,z)$ and time $t$
of a body in space.
\begin{enumerate}
\item[(a)] We examine a simplified version of the heat equation. Consider a straight wire "coordinatized" by $x$.
Then the temperature $T(x,t)$ at time $t$ and position $x$ along the wire is modeled by the one-dimensional heat equation
\[
	k\pdrr{T}{x} = \pdr{T}{t}.
\]
Show that the function $T(x,t) = e^{-kt}\cos{x}$ satisfies this equation. Note that if $t$ is held constant
at value $t_0$, then $T(x,t_0)$ shows how the temperature varies along the wire at time $t_0$. Graph the curves
$z=T(x,t_0)$ for $t_0=0,1,10$, and use them to understand the graph of the surface $z=T(x,t)$ for some
$t\geq 0$. Explain what happens to the temperature of the wire after a long period of time.
\end{enumerate}
\end{problem}

\begin{solution}
\vfill
\end{solution}
\newpage


\end{document}